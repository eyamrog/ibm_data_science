{
\import{diagrams/}{cover}
\begin{frame}
    \vspace{5cm}
    \textcolor{white}{Rogério Yamada \\ 22 September 2025}
\end{frame}
}

\begin{frame}{Outline}
    \tableofcontents
\end{frame}

\section{Executive Summary}

\begin{frame}{Executive Summary}

\end{frame}

\section{Introduction}

\begin{frame}{Introduction}

\end{frame}

\section{Methodology}

{
\import{diagrams/}{section_1}
\begin{frame}[plain]
\end{frame}
}

\begin{frame}{Methodology}
    \framesubtitle{Executive Summary}
    \begin{itemize}
        \item Data collection methodology
        \begin{itemize}
            \item Data about rocket launches was obtained from a SpaceX API and web scraping Wikipedia pages
        \end{itemize}
        \item Perform data wrangling
        \begin{itemize}
            \item Missing data was handled, a preliminary Exploratory Data Analysis was performed, and the variable \texttt{Outcome Class} was defined for training the supervised models
        \end{itemize}
        \item Perform Exploratory Data Analysis (EDA) using visualisation and SQL
        \item Perform interactive visual analysis using Folium and Plotly Dash
        \item Perform predictive analysis using classification models
    \end{itemize}
\end{frame}

\begin{frame}{Data Collection}
    \begin{itemize}
        \item SpaceX API data extraction and Wikipedia pages web scraping were combined to produce a dataset of SpaceX Falcon 9 landings information
    \end{itemize}
    \import{diagrams/}{data_collection_1.tex}
\end{frame}

{\nologo
\begin{frame}{Data Collection -- SpaceX API}
    \begin{columns}
        \column{0.3\textwidth}
            \begin{itemize}
                \item From the SpaceX API endpoint \url{https://api.spacexdata.com/v4/} we probed the following data sources:
                \begin{itemize}
                    \item rocket
                    \item payload
                    \item launchpad
                    \item cores
                \end{itemize}
                \item \href{https://github.com/eyamrog/ibm_data_science/blob/main/10_Data_Science_Capstone_Lab1_Collecting_the_Data.ipynb}{\uline{Jupyter Notebook's GitHub URL}}
            \end{itemize}
        \column{0.7\textwidth}
            \import{diagrams/}{data_collection_2}
    \end{columns}
\end{frame}
}

{\nologo
\begin{frame}{Data Collection -- Scraping}
    \begin{columns}
        \column{0.3\textwidth}
            \begin{itemize}
                \item From the Wikipedia \href{https://en.wikipedia.org/wiki/List_of_Falcon_9_and_Falcon_Heavy_launches}{\uline{List of Falcon 9 and Falcon Heavy launches}} web page we collected Falcon 9 historical launch records
                \item \href{https://github.com/eyamrog/ibm_data_science/blob/main/10_Data_Science_Capstone_Lab1_Collecting_the_Data.ipynb}{\uline{Jupyter Notebook's GitHub URL}}
            \end{itemize}
        \column{0.7\textwidth}
            \import{diagrams/}{data_collection_3}
    \end{columns}
\end{frame}
}

\begin{frame}{Data Wrangling}

\end{frame}

\begin{frame}{EDA with Data Visualisation}

\end{frame}

\begin{frame}{EDA with SQL}

\end{frame}

\begin{frame}{Build an Interactive Map with Folium}

\end{frame}

\begin{frame}{Build a Dashboard with Plotly Dash}

\end{frame}

\begin{frame}{Predictive Analysis (Classification)}

\end{frame}

\section{Results}

\begin{frame}{Results}
    \begin{itemize}
        \item Exploratory data analysis results
        \item Interactive analytics demo in screenshots
        \item Predictive analysis results
    \end{itemize}
\end{frame}

{
\import{diagrams/}{section_2}
\begin{frame}[plain]
\end{frame}
}

\begin{frame}{Flight Number versus Launch Site}

\end{frame}

\begin{frame}{Payload versus Launch Site}

\end{frame}

\begin{frame}{Success Rate versus Orbit Type}

\end{frame}

\begin{frame}{Flight Number versus Orbit Type}

\end{frame}

\begin{frame}{Payload versus Orbit Type}

\end{frame}

\begin{frame}{Launch Success Yearly Trend}

\end{frame}

\begin{frame}{All Launch Site Names}

\end{frame}

\begin{frame}{Launch Site Names Begin with `CCA'}

\end{frame}

\begin{frame}{Total Payload Mass}

\end{frame}

\begin{frame}{Average Payload Mass by F9 v1.1}

\end{frame}

\begin{frame}{First Successful Ground Landing Date}

\end{frame}

\begin{frame}{Successful Drone Ship Landing with Payload between 4000 and 60000}

\end{frame}

\begin{frame}{Total Number of Successful and Failure Mission Outcomes}

\end{frame}

\begin{frame}{Boosters Carried Maximum Payload}

\end{frame}

\begin{frame}{2015 Launch Records}

\end{frame}

\begin{frame}{Rank Landing Outcomes Between 2010-06-04 and 2017-03-20}

\end{frame}

{
\import{diagrams/}{section_3}
\begin{frame}[plain]
\end{frame}
}

\begin{frame}{<Folium Map Screenshot 1>}

\end{frame}

\begin{frame}{<Folium Map Screenshot 2>}

\end{frame}

\begin{frame}{Folium Map Screenshot 3}

\end{frame}

{
\import{diagrams/}{section_4}
\begin{frame}[plain]
\end{frame}
}

\begin{frame}{Dashboard Screenshot 1}

\end{frame}

\begin{frame}{Dashboard Screenshot 2}

\end{frame}

\begin{frame}{Dashboard Screenshot 3}

\end{frame}

{
\import{diagrams/}{section_5}
\begin{frame}[plain]
\end{frame}
}

\begin{frame}{Classification Accuracy}

\end{frame}

\begin{frame}{Confusion Matrix}

\end{frame}

\section{Conclusion}

\begin{frame}{Conclusions}
    \begin{itemize}
        \item
    \end{itemize}
\end{frame}

\section{Appendix}

\begin{frame}{Appendix}

\end{frame}

{
\import{diagrams/}{thank_you}
\begin{frame}[plain]
\end{frame}
}

%{\nologo
%\begin{frame}
%    \begin{columns}
%        \column{0.5\textwidth}
%            \import{diagrams/}{p_edgard_20250917_5}
%        \column{0.5\textwidth}
%            \import{diagrams/}{p_edgard_20250917_6}
%    \end{columns}
%\end{frame}
%}
