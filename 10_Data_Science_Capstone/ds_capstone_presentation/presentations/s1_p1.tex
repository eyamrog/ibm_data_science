{
\import{diagrams/}{cover}
\begin{frame}
    \vspace{5cm}
    \textcolor{white}{Rogério Yamada \\ 22 September 2025}
\end{frame}
}

\begin{frame}{Outline}
    \tableofcontents
\end{frame}

\section{Executive Summary}

\begin{frame}{Executive Summary}

\end{frame}

\section{Introduction}

\begin{frame}{Introduction}

\end{frame}

\section{Methodology}

{
\import{diagrams/}{section_1}
\begin{frame}[plain]
\end{frame}
}

\begin{frame}{Methodology}
    \framesubtitle{Executive Summary}
    \begin{itemize}
        \item Data collection methodology
        \begin{itemize}
            \item Data about rocket launches was obtained from a SpaceX API and web scraping Wikipedia pages
        \end{itemize}
        \item Perform data wrangling
        \begin{itemize}
            \item Missing data was handled, a preliminary Exploratory Data Analysis was performed, and the variable \texttt{Outcome Class} was defined for training the supervised models
        \end{itemize}
        \item Perform Exploratory Data Analysis (EDA) using visualisation and SQL
        \item Perform interactive visual analysis using Folium and Plotly Dash
        \item Perform predictive analysis using classification models
    \end{itemize}
\end{frame}

\begin{frame}{Data Collection}
    \begin{itemize}
        \item SpaceX API data extraction and Wikipedia pages web scraping were combined to produce a dataset of SpaceX Falcon 9 landings information
    \end{itemize}
    \import{diagrams/}{data_collection_1}
\end{frame}

{\nologo
\begin{frame}{Data Collection -- SpaceX API}
    \begin{columns}
        \column{0.3\textwidth}
            \begin{itemize}
                \item From the SpaceX API endpoint \url{https://api.spacexdata.com/v4/} we probed the following data sources:
                \begin{itemize}
                    \item rocket
                    \item payload
                    \item launchpad
                    \item cores
                \end{itemize}
                \item \href{https://github.com/eyamrog/ibm_data_science/blob/main/10_Data_Science_Capstone_Lab1_Collecting_the_Data.ipynb}{\uline{Jupyter Notebook's GitHub URL}}
            \end{itemize}
        \column{0.7\textwidth}
            \import{diagrams/}{data_collection_2}
    \end{columns}
\end{frame}
}

{\nologo
\begin{frame}{Data Collection -- Scraping}
    \begin{columns}
        \column{0.3\textwidth}
            \begin{itemize}
                \item From the Wikipedia \href{https://en.wikipedia.org/wiki/List_of_Falcon_9_and_Falcon_Heavy_launches}{\uline{List of Falcon 9 and Falcon Heavy launches}} web page we collected Falcon 9 historical launch records
                \item \href{https://github.com/eyamrog/ibm_data_science/blob/main/10_Data_Science_Capstone_Web_Scraping.ipynb}{\uline{Jupyter Notebook's GitHub URL}}
            \end{itemize}
        \column{0.7\textwidth}
            \import{diagrams/}{data_collection_3}
    \end{columns}
\end{frame}
}

{\nologo
\begin{frame}{Data Wrangling}
    \begin{columns}
        \column{0.3\textwidth}
            \begin{itemize}
                \item Through data wrangling, the variable \texttt{Outcome Class} was defined for training the supervised models
                \item \href{https://github.com/eyamrog/ibm_data_science/blob/main/10_Data_Science_Capstone_Lab2_Data_Wrangling.ipynb}{\uline{Jupyter Notebook's GitHub URL}}
            \end{itemize}
        \column{0.7\textwidth}
            \import{diagrams/}{data_wrangling}
    \end{columns}
\end{frame}
}

\begin{frame}{EDA with SQL}
    \begin{itemize}
        \item Several SQL queries have been processed to gain insights about the landing outcomes:
        \begin{itemize}
            \item Launching sites
            \item Total payload mass carried by specific boosters in specific sites
            \item Successful and failed landing outcomes
        \end{itemize}
        \item \href{https://github.com/eyamrog/ibm_data_science/blob/main/10_Data_Science_Capstone_SQL_Notebook_for_Peer_Assignment.ipynb}{\uline{Jupyter Notebook's GitHub URL}}
    \end{itemize}
\end{frame}

\begin{frame}{EDA with Data Visualisation}
    \begin{itemize}
        \item Several charts have been drawn to gain insights about the landing outcomes:
        \begin{itemize}
            \item Flight Number versus Launch Site by Class
            \item Payload Mass versus Launch Site by Class
            \item Success Rate by Orbit
            \item Flight Number versus Orbit by Class
            \item Payload Mass versus Orbit by Class
            \item Yearly Launch Success Rate
        \end{itemize}
        \item \href{https://github.com/eyamrog/ibm_data_science/blob/main/10_Data_Science_Capstone_EDA_with_Visualisation.ipynb}{\uline{Jupyter Notebook's GitHub URL}}
    \end{itemize}
\end{frame}

\begin{frame}{Build an Interactive Map with Folium}
    \begin{itemize}
        \item A geographical analysis has been performed to gain insights about the dependencies of landing outcomes and location and surrounding of launching sites:
        \begin{itemize}
            \item Marking all launch sites on a map
            \item Marking the success/failed launches for each site on the map
            \item Calculating the distances between a launch site to its proximities
        \end{itemize}
        \item \href{https://github.com/eyamrog/ibm_data_science/blob/main/10_Data_Science_Capstone_Interactive_Visual_Analytics_with_Folium.ipynb}{\uline{Jupyter Notebook's GitHub URL}}
    \end{itemize}
\end{frame}

\begin{frame}{Build a Dashboard with Plotly Dash}
    \begin{itemize}
        \item A dashboard has been implemented to perform real-time analysis about landing outcomes considering:
        \begin{itemize}
            \item Launch site drop-down menu
            \item Interactive successful landing outcome pie chart
            \item Range slider for selecting payload mass
            \item Interactive successful landing outcome scatter plot
        \end{itemize}
        \item \href{https://github.com/eyamrog/ibm_data_science/blob/main/spacex_dash_app.ipynb}{\uline{Jupyter Notebook's GitHub URL}}
    \end{itemize}
\end{frame}

\begin{frame}{Predictive Analysis (Classification)}
    \begin{columns}
        \column{0.3\textwidth}
            \begin{itemize}
                \item The following predictive models have been considered:
                \begin{itemize}
                    \item Logistics Regression
                    \item Support Vector Machine
                    \item Decision Tree
                    \item K Nearest Neighbours
                \end{itemize}
                \item \href{https://github.com/eyamrog/ibm_data_science/blob/main/10_Data_Science_Capstone_Machine_Learning_Prediction.ipynb}{\uline{Jupyter Notebook's GitHub URL}}
            \end{itemize}
        \column{0.7\textwidth}
            \import{diagrams/}{predictive_analysis}
    \end{columns}
\end{frame}

\section{Results}

\begin{frame}{Results}
    \begin{itemize}
        \item Exploratory data analysis results
        \item Interactive analytics demo in screenshots
        \item Predictive analysis results
    \end{itemize}
\end{frame}

{
\import{diagrams/}{section_2}
\begin{frame}[plain]
\end{frame}
}

\begin{frame}{Flight Number versus Launch Site}
    \import{diagrams/}{flight_number_versus_launch_site_by_class}
    \begin{itemize}
        \item There is a concentration of unsuccessful landing outcomes in the site CCAFS SLC 40
        \item As the number of flights increase, the successful landing outcomes are likely to increase
    \end{itemize}
\end{frame}

\begin{frame}{Payload versus Launch Site}
    \import{diagrams/}{payload_mass_versus_launch_site_by_class}
    \begin{itemize}
        \item There is a concentration of unsuccessful landing outcomes in CCAFS SLC 40 for payloads lighter than 7000 kg
    \end{itemize}
\end{frame}

\begin{frame}{Success Rate versus Orbit Type}
    \import{diagrams/}{success_rate_by_orbit}
    \begin{itemize}
        \item The success rate is higher when the orbit type is ES-L1, GEO, HEO, and SSO
        \item The success rate is lower when the orbit type is GTO and SO
    \end{itemize}
\end{frame}

\begin{frame}{Flight Number versus Orbit Type}
    \import{diagrams/}{flight_number_versus_orbit_by_class}
    \begin{itemize}
        \item The success rate is higher as the number of flights increases for orbit type LEO, ISS, and PO
        \item There seems to be no relationship between flight number when the orbit type is GTO
    \end{itemize}
\end{frame}

\begin{frame}{Payload versus Orbit Type}
    \import{diagrams/}{payload_mass_versus_orbit_by_mass}
    \begin{itemize}
        \item With heavy payloads, the successful landing rate is higher for PO, LEO, and ISS orbit types
        \item However, regarding GTO, we cannot distinguish this well as both positive landing rate and negative landing (unsuccessful mission) occur almost evenly
    \end{itemize}
\end{frame}

\begin{frame}{Launch Success Yearly Trend}
    \import{diagrams/}{yearly_launch_success_rate}
    \begin{itemize}
        \item The success rate since 2013 kept increasing until 2017 (stable in 2014) and after 2015 it started increasing
    \end{itemize}
\end{frame}

\begin{frame}{All Launch Site Names}
    \import{diagrams/}{all_launch_site_names}
    \begin{itemize}
        \item This is a sorted list of unique, non-null values from the Launch\_Site column in the SPACEXTABLE table
    \end{itemize}
\end{frame}

\begin{frame}{Launch Site Names Begin with `CCA'}
    \import{diagrams/}{launch_site_names_begin_with_cca}
    \begin{itemize}
        \item This is a set of up to 5 rows from the SPACEXTABLE table where the Launch\_Site column starts with the text ``CCA'' (e.g., ``CCAFS \dots''). The LIKE `CCA\%' filter matches any value beginning with ``CCA''
    \end{itemize}
\end{frame}

\begin{frame}{Total Payload Mass}
    Total Payload Mass = 45596
    \begin{itemize}
        \item This is the total payload mass (sum of PAYLOAD\_MASS\_\_KG\_) for all rows in SPACEXTABLE where the Customer is exactly `NASA (CRS)'. The result is a single value labeled total\_payload\_mass
    \end{itemize}
\end{frame}

\begin{frame}{Average Payload Mass by F9 v1.1}
    Average Payload Mass = 2534.6666666666665
    \begin{itemize}
        \item This is the average payload mass (PAYLOAD\_MASS\_\_KG\_) over rows in SPACEXTABLE whose Booster\_Version starts with ``F9 v1.1'', returning a single value named avg\_payload\_mass
    \end{itemize}
\end{frame}

\begin{frame}{First Successful Ground Landing Date}
    First Ground Pad Landing Success Date = 2015-12-22
    \begin{itemize}
        \item This is the earliest Date (minimum) among rows where Landing\_Outcome is exactly ``Success (ground pad)'', labeled first\_ground\_pad\_success\_date.
    \end{itemize}
\end{frame}

\begin{frame}{Successful Drone Ship Landing with Payload between 4000 and 60000}
    \import{diagrams/}{booster_name}
    \begin{itemize}
        \item These are the unique booster versions (as booster\_name) that had a ``Success (drone ship)'' landing and carried payloads strictly between 4000 kg and 6000 kg, sorted alphabetically
    \end{itemize}
\end{frame}

\begin{frame}{Total Number of Successful and Failure Mission Outcomes}
    \import{diagrams/}{total_number_of_successful_and_failure_mission_outcomes}
    \begin{itemize}
        \item These are rows grouped by Mission\_Outcome, counting how many are successes or failures (strings starting with ``Success'' or ``Failure''), and returns each outcome with its count, sorted by count descending
    \end{itemize}
\end{frame}

\begin{frame}{Boosters Carried Maximum Payload}
    \import{diagrams/}{booster_version}
    \begin{itemize}
        \item These are the unique booster versions that carried the maximum payload mass found in the table. The subquery gets the global MAX(PAYLOAD\_MASS\_\_KG\_), and the outer query lists distinct Booster\_Version rows matching that value
    \end{itemize}
\end{frame}

\begin{frame}{2015 Launch Records}
    \import{diagrams/}{2015_launch_records}
    \begin{itemize}
        \item This is a list of 2015 records where Landing\_Outcome is ``Failure (drone ship)''
    \end{itemize}
\end{frame}

\begin{frame}{Rank Landing Outcomes Between 2010-06-04 and 2017-03-20}
    \import{diagrams/}{rank_landing_outcomes}
    \begin{itemize}
        \item This is a calculation of how many launches fall into each Landing\_Outcome between 2010-06-04 and 2017-03-20, returning each outcome with its count, sorted by count descending
    \end{itemize}
\end{frame}

{
\import{diagrams/}{section_3}
\begin{frame}[plain]
\end{frame}
}

\begin{frame}{All launch sites map}
    \import{diagrams/}{map_all_launch_sites}
    \begin{itemize}
        \item All launch sites are plotted on the map
    \end{itemize}
\end{frame}

\begin{frame}{Success/Failed launches}
    \import{diagrams/}{map_success_failed_launches}
    \begin{itemize}
        \item The successful and failed launches are clustered on the map
    \end{itemize}
\end{frame}

\begin{frame}{Coastline point map}
    \import{diagrams/}{map_coastline_point}
    \begin{itemize}
        \item There a distance of 0.51 km between the CCAFS SLC-40 launch site and the coastline
    \end{itemize}
\end{frame}

{
\import{diagrams/}{section_4}
\begin{frame}[plain]
\end{frame}
}

\begin{frame}{Launch site drop-down menu}
    \import{diagrams/}{launch_site_drop-down_menu}
    \begin{itemize}
        \item Launch site drop-down menu
    \end{itemize}
\end{frame}

\begin{frame}{Interactive successful landing outcome pie chart}
    \import{diagrams/}{interactive_successful_landing_outcome_pie_chart}
    \begin{itemize}
        \item Interactive successful landing outcome pie chart
    \end{itemize}
\end{frame}

\begin{frame}{Complete Dashboard}
    \begin{columns}
        \column{0.5\textwidth}
            \begin{itemize}
                \item This is the complete dashboard with the following elements:
                \begin{itemize}
                    \item Launch site drop-down menu
                    \item Interactive successful landing outcome pie chart
                    \item Range slider for selecting payload mass
                    \item Interactive successful landing outcome scatter plot
                \end{itemize}
            \end{itemize}
        \column{0.5\textwidth}
            \import{diagrams/}{complete_dashboard}
    \end{columns}
\end{frame}

{
\import{diagrams/}{section_5}
\begin{frame}[plain]
\end{frame}
}

\begin{frame}{Classification Accuracy}

\end{frame}

\begin{frame}{Confusion Matrix}

\end{frame}

\section{Conclusion}

\begin{frame}{Conclusions}
    \begin{itemize}
        \item
    \end{itemize}
\end{frame}

\section{Appendix}

\begin{frame}{Appendix}

\end{frame}

{
\import{diagrams/}{thank_you}
\begin{frame}[plain]
\end{frame}
}

%{\nologo
%\begin{frame}
%    \begin{columns}
%        \column{0.5\textwidth}
%            \import{diagrams/}{p_edgard_20250917_5}
%        \column{0.5\textwidth}
%            \import{diagrams/}{p_edgard_20250917_6}
%    \end{columns}
%\end{frame}
%}
